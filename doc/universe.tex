\documentclass[a4paper]{article}

\usepackage[english]{babel}
\usepackage[utf8]{inputenc}
\usepackage{amsmath}

\usepackage{fancyhdr}
\makeatletter
\usepackage{graphicx}

\usepackage[colorinlistoftodos]{todonotes}
\usepackage{tabulary}
\usepackage{textcomp}
% http://tex.stackexchange.com/questions/26462/make-a-table-span-multiple-pages
\usepackage{longtable}
\usepackage{listings}


\usepackage[mathscr]{euscript}
% \mathfrak{M} 
% \usepackage{ amssymb }

\newcommand*{\backin}{\rotatebox[origin=c]{-180}{$\in$}}%

\title{Universe}

\author{Alexander Loshchilov}

\date{2016-05-18}

\begin{document}

\maketitle



\newpage

\tableofcontents

\newpage

\section{Concept}

A hypothetic \textit{\textbf{Universe}}, where in its 2D Space, \textit{\textbf{Forces}} control what happen. There is \textit{\textbf{Matter}} build out of different \textit{\textbf{Atoms}} able to move. Each \textit{\textbf{Atom}} affect the \textit{\textbf{Forces}}. The \textit{\textbf{Natural Laws}} define how \textit{\textbf{Atoms}} convert into other \textit{\textbf{Atoms}}, if certain \textit{\textbf{Forces}} are present. We interact with \textit{\textbf{Universe}} using \textit{\textbf{Agents}}, which perceive \textit{\textbf{Forces}} and \textit{\textbf{Matters}}.

\newpage

\section{Mathematical model} 

So far our discussion about notions involded in universe has been rather imprecise.
Let us formally define a \textit{\textbf{Universe}} as quintuple
$ (\mathscr{M}, \mathscr{A}, \mathscr{F}, \mathscr{N}, \mathscr{I}) $,
where $ \mathscr{M} $ is a set of \textit{\textbf{Matters}}, $ \mathscr{A} $ is a set of \textit{\textbf{Atoms}},
$ \mathscr{F} $ is a set of \textit{\textbf{Forces}}, $ \mathscr{N} $ is a set of \textit{\textbf{Natural Laws}} and
$ \mathscr{I} $ is a set of $ Agents $.

\textit{\textbf{Matter}} is build out of different \textit{\textbf{Atoms}}.

$\renewcommand{\arraystretch}{1.3}
\begin{array}[c]{ccccc}
\varphi:&\mathscr{M}&\rightarrow&\mathscr{P}(X)\\
&\rotatebox[origin=c]{90}{{$\in$}}&&\rotatebox{90}{$\in$}&\\
&m&\mapsto& A &\subseteq \mathscr{A}
\end{array}$

\end{document}
